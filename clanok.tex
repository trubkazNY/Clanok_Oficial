% Metódy inžinierskej práce

\documentclass[10pt,twoside,slovak,a4paper]{article}

\usepackage[slovak]{babel}
%\usepackage[T1]{fontenc}
\usepackage[IL2]{fontenc} % lepšia sadzba písmena Ľ než v T1
\usepackage[utf8]{inputenc}
\usepackage{graphicx}
\usepackage{url} % príkaz \url na formátovanie URL
\usepackage{hyperref} % odkazy v texte budú aktívne (pri niektorých triedach dokumentov spôsobuje posun textu)

\usepackage{cite}
%\usepackage{times}

\pagestyle{headings}

\title{Techniky spracovanie veľkých dát\thanks{Semestrálny projekt v predmete Metódy inžinierskej práce, ak. rok 2023/24, vedenie: Vladimír Mlynarovič}} % meno a priezvisko vyučujúceho na cvičeniach

\author{Tomáš Zenka\\[2pt]
	{\small Slovenská technická univerzita v Bratislave}\\
	{\small Fakulta informatiky a informačných technológií}\\
	{\small \texttt{xzenka@stuba.sk}}
	}

\date{\small 5. november 2023} % upravte



\begin{document}

\maketitle

\begin{abstract}
Článok skúma a porovnáva techniky spracovania veľkého množstva dát, čo je kľúčové v dnešnej digitálnej dobe, kde sa generuje obrovské množstvo informácií. Cieľom je poskytnúť prehľad o moderných prístupoch a nástrojoch určených na manipuláciu s masívnymi dátovými súbormi. Tieto nástroje zahŕňajú distribuované systémy na spracovanie dát, algoritmy strojového učenia a metriky na hodnotenie kvality dát. Dôraz sa kladie na potrebu rýchleho spracovania dát v reálnom čase, čo umožňuje rýchlu analýzu a tvorbu hodnotných poznatkov z týchto objemných dátových zdrojov. Článok taktiež uvádza príklady aplikácií v rôznych odvetviach, ako je medicína, finančníctvo a priemysel. Spracovanie veľkého množstva dát sa stáva nevyhnutným nástrojom pre konkurencieschopnosť a inovácie v súčasnom digitálnom prostredí.
\end{abstract}



\section{Úvod}

Zatial len v odrážkach
\begin{itemize}
\item úvod do problematiky
\item definícia veľkých dát
\item nárast objemu dát
\item výzvy spojené so spracovaním veľkých dát
\item cieľ a štruktúra článku
\end{itemize}



\section{Základné techniky spracovania veľkých dát} \label{techniky}

Stručný popis (1 - 2 vety) na techniky, ktoré možme použiť.
\begin{enumerate}
\item NoSQL databázy \ref{NoSQL}
\item In-memory processing \ref{In-memory processing}
\end{enumerate}

\subsection{NoSQL databázy} \label{NoSQL}

\subsection{In-memory processing} \label{In-memory processing}

\section{Iná časť} \label{ina}

Základným problémom je teda\ldots{} Najprv sa pozrieme na nejaké vysvetlenie (časť~\ref{ina:nejake}), a potom na ešte nejaké (časť~\ref{ina:nejake}).\footnote{Niekedy môžete potrebovať aj poznámku pod čiarou.}

Môže sa zdať, že problém vlastne nejestvuje\cite{Coplien:MPD}, ale bolo dokázané, že to tak nie je~\cite{Czarnecki:Staged, Czarnecki:Progress}. Napriek tomu, aj dnes na webe narazíme na všelijaké pochybné názory\cite{PLP-Framework}. Dôležité veci možno \emph{zdôrazniť kurzívou}.


\subsection{Nejaké vysvetlenie} \label{ina:nejake}

Niekedy treba uviesť zoznam:

\begin{itemize}
\item jedna vec
\item druhá vec
	\begin{itemize}
	\item x
	\item y
	\end{itemize}
\end{itemize}

Ten istý zoznam, len číslovaný:

\begin{enumerate}
\item jedna vec
\item druhá vec
	\begin{enumerate}
	\item x
	\item y
	\end{enumerate}
\end{enumerate}


\subsection{Ešte nejaké vysvetlenie} \label{ina:este}

\paragraph{Veľmi dôležitá poznámka.}
Niekedy je potrebné nadpisom označiť odsek. Text pokračuje hneď za nadpisom.



\section{Dôležitá časť} \label{dolezita}




\section{Ešte dôležitejšia časť} \label{dolezitejsia}




\section{Záver} \label{zaver} % prípadne iný variant názvu



%\acknowledgement{Ak niekomu chcete poďakovať\ldots}


% týmto sa generuje zoznam literatúry z obsahu súboru literatura.bib podľa toho, na čo sa v článku odkazujete
\bibliography{literatura}
\bibliographystyle{plain} % prípadne alpha, abbrv alebo hociktorý iný
\end{document}
